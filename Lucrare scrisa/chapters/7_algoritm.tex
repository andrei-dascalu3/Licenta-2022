\chapter{Algoritmul de repartizare}

Problema pe care își propune această aplicație să o rezolve este o instanță a problemei asignării sau problema atribuirii (\textit{assignment problem}). Este de precizat că cele două nume vor fi folosite interschimbabil în secțiunile următoare. În cazul de față, instanța este reprezentată de o mulțime de studenți, fiecare cu o ordonare a unor prefereințe, și o mulțime a propunerilor profesorilor (proiecte sau tematici generale). Cerința este reapartizarea optimă a propunerilor către studenți în funcție de preferințele acestora.

În literatura de specialitatea au fost descriși și dezvoltați diverși algoritmi de rezolvare a unei astfel de instanțe a problemei atribuirii. Algoritmul implementat de aplicația \thesistitle{} este unul din clasa de algoritmi de licitație (\textit{auction algorithms}), descris în lucrarea "Assignment Problem with Constraints" scrisă de către Ulrich Bauer, Facultatea de Informatică a Universității Tehcnice din München, 2005. Pentru a descrie coerent și cât mai clar logica acestui algoritm, în secțiunile următoare vor fi descrise problema atribuirii și similitudinile acesteia cu o problemă de flux, conceptele matematice ale algoritmilor propuși pentru rezolvare și pașii algoritmului implementat pentru instanța prezentă.

\section{Problema asignării (atribuirii)}

Problema simetrică a asignării constă în două mulțimi $X$ și $Y$ de dimensiuni egale, o mulțime $E \subseteq X \times Y$ și o funcție de cost $c_{xy}$ pentru oricare pereche posibilă $(x, y) \in E$. Scopul este cuplarea oricărui element din $X$ cu un element din $Y$ ca la final costul total să fie minim \cite{assignment}.

În contextul grafurilor, această problemă poate fi redusă la problema fluxului de cost minim într-un  graf bipartit $G=((X \bigcup Y,\ E)$ cu o funcție de cost $c_{xy},\ (x, y) \in E$, și capacitatea $u_xy = 1, \forall (x, y) \in E$. 

Expresia matematică a problemei \cite[p.~6]{assignment} este 
\[ minimizarea \sum_{(i, j) \in E} c_{ij} x_{ij} \]
astfel încât
\[ \sum_{j:(i, j) \in E} x_{ij} = 1, \forall i \in X, \]
\[ \sum_{i:(i, j) \in E} x_{ij} = 1, \forall j \in Y, \]
\[ x_{ij} \geq 0, \forall (i, j) \in E \].

Variabila $x_{ij}$ indică câtă unități de flux sunt trimise pe muchia $(i, j)$.

Cu toate acestea, problema repartizării fiecărui student o teză de licență în funcție de preferințele sale este o problemă asimetrică deoarece numărul de propuneri, $|Y|$, este mai mare sau egal decât numărul de studenți (trebuie să fie măcar egal pentru a putea atribui fiecărui student o lucrare). La final, o soluția determinată se va afla în variabila $x_{ij}$ unde dacă $x_{ij} = 1$, atunci studentului $i$ îi este repartizată propunerea $j$ din totalul de propuneri al tuturor profesorilor.

\section{Algoritmul}

\section{Descrierea particularităților}


\section{Implementare}

