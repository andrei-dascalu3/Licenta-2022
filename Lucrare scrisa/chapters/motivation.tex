\chapter*{Motivație} 
\addcontentsline{toc}{chapter}{Motivație}

Lucrarea de licență reprezintă culminarea anilor de facultate, o dovadă a studentului de deprindere a unor noțiuni, de specializare într-un anumit domeniu și de capabilitatea a acestuia de a contribui cu soluții proprii la problemele curente sau viitoare ale societății. În consecință, alegerea unei tematici adecvate pentru lucrarea de licență este un pas de bază datorită unor motive pertinente.
În primul rând, abordarea unei tematici cât mai aproape de domeniile sau subiectele de interes ale studentului rezultă într-o atenție mai mare și o implicare corespunzătoare ale acestuia în realizarea în sine a lucrării.
În al doilea rând, există o legătură directă între eficiența și temeinicia realizări lucrării de licență și colaborarea dintre profesorul coordonator și student, fiind necesară o comunicare constantă și productivă prind feedback-ul în ambele sensuri, dar și prin resurse, materiale sau idei.

Profesorii coordonatori au un număr de locuri limitat, prin urmare este necesară o repartizare eficientă și mai ales optimă a studenților în funcție de preferințe, luând în același timp în calcul și punctele de vedere ale profesorilor. De asemenea, studenții ar trebui să aibă posibilitatea să cunoască dinainte tematicile propuse de profesori și eventualele condiții prealabile de realizare a respectivelor teme. O aplicație centralizată este așadar întrutotul necesară optimizării și îmbunătățirii în general ale acestui proces organizatoric din cadrul facultății.