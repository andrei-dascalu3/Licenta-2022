\chapter*{Concluzii} 
\addcontentsline{toc}{chapter}{Concluzii}

Aplicația \thesistitle propune să rezolve problema des întâlnită în facultăți de a repartiza studenții la profesorii de licență, mai precis de a atribui tezele propuse studenților ținând cont de preferințele lor. Într-un final, scopul principal este de a oferi șansa viitorilor absolvenți de a realiza o lucrare finală pe măsura aptitudinilor cumulate pe parcursul anilor de studenție, dar și în funcție de înclinațiile lor spre o anumită specializare. Mai mult decât atât, aplicația oferă posibilitatea unei cercetări riguroase de către student a proiectelor propuse prin centralizarea acestora și opțiunea de a conveni cu un profesor asupra unei propuneri favorabile, sub forma unui acord.

Dezvoltarea acestei aplicații a oferit prilejul cercetării mai amănunțite a tehnologiilor utilizate și dezvoltării abilităților în crearea de soluții software. Aplicația prezintă astfel interes în modul de adaptare a interfeței în funcție de utilizator și în același timp adoptarea unui design simplist și intuitiv. Partea de back-end prezintă o implementare personalizată a mecanismului de securitate adaptat unui context special, de uz intern.

Pe lângă aceste aspecte, urmând a rezolva problema de față, au fost cercetați mai mulți algoritmi potențiali, alegându-se într-un final un algoritm de atribuire prin double-push cu heap. Una dintre provocări a fost adaptarea datelor problemei la cele de intrare (input) și optimizarea prelucrării informațiilor de la baza de date. De asemenea, implementarea conceptelor teoretice descrise în Java constă un alt interes prin modularizarea și particularitățile codului.

\section*{Posibile dezvoltări viitoare}

Aplicația prezintă o serie de direcții de dezvoltare în viitor, prezentând un potențial de a deveni după anumite îmbunătățiri și modificări o aplicație utilitară robustă în contextul intern al facultății.

O primă idee de dezvoltare este implementarea unor noi algoritmi adaptați datelor de intrare în urma cercetării performanțelor acestora. O altă idee este implementarea unui sistem de preferințe și pentru profesori care vor putea nu numai încheia acorduri pentru anumite lucrări propuse, ci și de a crea o listă ierarhizată de studenți potențial coordonați. Studenți la rândul lor ar putea avea opțiunea de a propune o lucrare. De asemenea, introducerea de noi utilizatori ar putea fi mult simplificată prin intermediul încărcării și procesării unui fișier \texttt{xml} sau \texttt{csv}. În final, hostarea acestei aplicații utlizând servicii cloud precum cele ale Google sau Amazon conferă scalabilitate și accesibilitate.