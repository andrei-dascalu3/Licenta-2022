\chapter{Back-end}

Partea de back-end a aplicației \thesistitle{} a fost implementată în framework-ul de Java Spring Boot. Această tehnologie în primul rând oferă o varietate de utilități pentru dezvoltarea web, procesare în paralel, tranazacții cu baza de date \cite{spring-boot-pros-cons}. Acesta beneficiază de un server integrat, în cazul de față este vorba de Tomcat, permițând un eventual \textit{deployment} mai ușor al aplicației. De asemenea, spre deosebire de Spring, Spring Boot nu are nevoie de o configurare XML.

În al doilea rând, acest framework permite accesul la o serie utilă de plugin-uri care permit dezvoltarea unei securități adecvate, comunicare facilă cu baza de date și simplificarea codului

Ultimul argument pentru alegerea acestei tehnologii este familiaritatea cu acesta și comunitatea extinsă de utilizatori și tutoriale.

Codul aplicației de back-end a \thesistitle{} este structurat pe straturi (layers) ce permite localizarea ușoară a claselor în funcție de rolurile acestora \cite{spring-boot-code-structure}. Există astfel module pentru \textit{controllers}, \textit{models}, \textit{repository}, \textit{service}, \textit{algorithm} etc.

\section{Inițializarea proiectului}



\section{Configurarea}

Fișierul \texttt{application.properties} menționat în secțiunea \textbf{4.6 Generarea datelor} conține elemente de configurare a aplicației, în special referitor la conexiunea cu baza de date. Este setat tipul bazei de date, în acest caz este MySQL, adresa la care rulează baza de date, username-ul și parola pentru conectare, precum și dialectul. Există 

\section{Modelele}

Pentru a simula înregistrările dintr-o bază de date, Spring Boot dispune de posibilitatea de a defini clase ce identifică anumite entități cu ajutorul unor adnotări specifice. Spring Boot utilizează \textit{Entity Scanning} pentru a le identifica în loc de un fișier special cum ar fi \texttt{persistence.xml} în Spring. Clasele luate în considerare sunt cele adnotate cu \texttt{Entity}, \texttt{Embeddable} sau \texttt{MappedSuperclass} \cite{spring-boot-docs}.

\section{Autentificare}


\section{Abordare}

